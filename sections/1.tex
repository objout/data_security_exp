\documentclass[../main.tex]{subfiles}
\graphicspath{{\subfix{../figures/}}}

\begin{document}
\chapter{PGP 软件应用实验}
\section{实验介绍}
PGP 软件是一款优秀的个人安全防护软件。请安装并使用该软件的主要功能,
进行相关实验,验证 PGP 软件的可用性和有效性。
%
\section{实验内容}
密钥对的产生、公钥的导出导入、文件的加密、Email 的加密和签名、文件
的粉碎、虚拟磁盘加密、磁盘空间的粉碎等功能。
%
\section{报告内容}
按功能模块进行实验,并组织书写实验步骤与实验结果,
分别以不同小节给出。实验中请使用与学号、姓名等特征信息相关的实验数据,
体现相关实验是自己所完成的。
% 具体书写建议:每个部分先给出功能性简介描述;再针对每个功能模块进行实验操作,
% 给出相关步骤的文字描述,并附上相应步骤和结果贴图。贴图要大小合适,截图清晰。
% 建议按照以下几个部分书写,不能完成的部分就删除。
% 相关的技术原理等可以适当进行一些描述,以流程图、文字结合方式为佳。
\section{PGP 密钥对的产生与管理}
% 该部分包括个人密钥对的生成、公钥的导出与导入等。

\section{文件的加密与签名}
% 该部分包括使用 PGP 对文件、文件夹的加密、加密\&签名、签名等功能的操作。

\section{Email 的加密与签名}
% 该部分包括使用使用 Web 网页的 email 的书写、剪切、加密剪贴板内容、邮件发送、
% 邮件接收、复制密文邮件、PGP 解密剪贴板内容等步骤。

% 加密邮件可以 2 个同学以上合作完成一个实验,分别书写个人通信、操作的部分。

\section{文件的粉碎}
% 该部分包括使用 PGP 软件安全删除粉碎个人的隐私数据文档、文件夹等。

\section{虚拟磁盘的加密}
% 该部分包括使用 PGP 创建、加密生成一个加密的虚拟磁盘功能。

\section{磁盘空间的粉碎}
% 该部分包括使用 PGP 右击逻辑磁盘,实现对磁盘已删除内容的彻底清空操作。

%[1] 中 国 互 联 网 协 会 2006 年 第 一 次 反 垃 圾 邮 件 调 查 结 果
%[EOL].http://www.anti-spam.cnlShowArticle.php?id=2713
%[2] 曹麒麟,张千里.垃圾邮件与反垃圾邮件技术[M].北京:人民邮电出版社,2003.
%[3] 罗改龙.基于 SPI 的防火墙的研究与实现[硕士学位论文].武汉:武汉理工大学,2007
%[4] 周茜,赵明生.中文文本分类中的特征选择研究[J].中文信息学报,2003,Vol.18 No.3
\end{document}
