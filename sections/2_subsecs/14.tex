\subsection{实验目的-分析 oracle 日志}
分析 oracle 日志.
%
\subsection{实验原理}
%
\subsection{实验环境}
\begin{itemize}
  \item Windows server 2003:192.168.1.2 wireshark、orabrute、sqlplus
  \item Windows server 2008 R2:192.168.1.3oracle 11g
\end{itemize}
%
\subsection{实验步骤}
【实验原理】
1) TNS 协议
TNS 协议是 ORACLE 服务端和客户端通讯的协议。TNS 协议传输可以使用 TCP/IP 协议、使用 SSL
的 TCP/IP 协议、命名管道和 IPC 协议传输,其中 TCP/IP 协议传输是使用明文传送。 ORACLE
网络通讯协议 TNS 有许多版本,并且大部分向下兼容。
TNS 协议有一个通用的头,通用头包含一个请求数据类型。不同的服务请求和数据,传输使用不
同的请求数据类型。Type 标识数据包的类型 ,说明如下:
类型号 类型说明
1 连接(CONNECT)
2 接受(ACCEPT)
3 确认(ACK)
4 拒绝(REFUTE)
5 重定向(REDIRECT)
6 数据(DATA)
7 NULL
8
9 中止(ABORT)
10
11 重新发送(RESEND)
12 标记(MARKER)
13 ATTENTION
14 控制(CONTROL)
2) oracle 加密原理
当 Oracle 发起连接后,Oracle 客户端向 oracle 数据库发送自己版本号、包含的加密算法等信
息。最终服务端确定使用什么加密算法,然后进行 O3logon 验证。O3logon 验证是一种查询-响
应协议,它利用 DES 加密技术保护这个会话的密钥(sesskey),保证 sesskey 不会在网络中传
输,所以即使有人监听网络也不会暴露核心密钥。其中 O3logon 验证的核心是 sesskey。
Oracle 11g 在 10g 的基础上进行了一定的改变。假设我们已经取得一个含有 Oracle 登录信息的
网络通讯包。省略掉前面和密码关系不大的信息在数据包中寻找到 4 个相关信息,分别是数据库
发送给客户端的 S_AUTH_SESSKEY、AUTH_VFR_DATA、客户端发送给服务器的 C_AUTH_SESSKEY 和 A
UTH_PASSWORD。
假设取得了 Oracle_hash,11g 基本同于 10g,客户端和数据库分别以 Oracle_hash 为基础生成 S
_AUTH_SESSKEY 和 C_AUTH_SESSKEY。客户端对传过来的 S_AUTH_SESSKEY 做 AES192 解密处理拿到
server_sesskey,把 server_sesskey 和自己的 client_sesskey 做 md5 生成 combine,用 combi
ne 生成 AUTH_PASSWORD。服务器端最后用 combine 对 AUTH_PASSWORD 解密,对比密码,如果密码
一致则登陆成功。
11g 最大的变化在生成 Oracle_hash 上采取了和 10g 不同的策略。Oracle 11g 为了提高 Oracle_
hash 的安全性,引入了 AUTH_VFR_DATA 这个随机值,取消了明文密码。每个会话的 AUTH_VFR_D
ATA 都不同。从根本上避免 9i、10g 同字符串(用户名+密码组成的字符串)带来的无论哪台机
器 oracle_hash 一致的巨大安全隐患。
从 11g 开始,oracle 和密码相关登陆信息全部采用了密文。有效地加大了破解难度。对于 orac
le 安全性问题,一定注意防止网络监听,设计 SID 的时候尽量避免 ORCL、TEST 等常用名。端口
号尽量不要选用 1521 和 1523 来增加扫描难度。使用复杂密码,定期更换密码等都会有助于 o
racle 的安全
3) Orabrute
这里我们使用 Orabrute 工具来进行远程破解 oracle,在使用这个工具的时候,需要系统提
前安装好 sqlplus,该工具的 原理很简单,就是不停的调用 sqlplus 然后进行登录验证,
帐户选择的是 sys,密码则为 password.txt 中的密 码单词。只要登录成功 ,就会调用 s
electpassword.sql 脚本抓取出在 SYS.USER$表中的其他用户的哈希值,然后退出程序。这里
有个值得注意的地方,当第二次运行 Orabrute 的时候,需要删除或移动同目录下的前一次运
行 Orabrute 时生成的 thepasswordsarehere .txt 文件。
4) SQLPLUS
Oracle 的 sqlplus 是与 oracle 数据库进行交互的客户端工具,借助 sqlplus 可以查看、修改数
据库记录。在 sqlplus 中,可以运行 sqlplus 命令与 sql 语句。
【实验环境】
【实验步骤】
一、破解 oracle 用户 sys 的密码,并使用 wireshark 抓取破解数据包
1.1 为了更直观地查看日志,在 Windows server 2008 R2 上,单击“开始”“管理工具”“事
件查看器”,展开“Windows 日志”,右键“应用程序”,选择清除日志。如图 1 所示
图 1
1.2 在 Windows server 2003(192.168.1.2)上打开 wireshark,选择要监听的网卡后,单击“S
tart”按钮,开始进行抓包。如图 2 所示
图 1
1.3 事先先删除桌面上 tools\oracle 目录下的 thepasswordsarehere .txt 文件,没有则跳
过。
1.4 打开 cmd,输入命令“cd C:\Documents and Settings\Administrator\桌面\tools\oracle”,
再输入命令“orabrute 192.168.1.3 1521 orcl 10 |more”,这里为了便于查看破解
信息,加了“|more”,在破解过程中,可安空格键查看之后的内容。如图 3 所示
图 3
1.5 除了在 cmd 中可以查看到破解结果,还可以在 thepasswordsare.txt 文件中进行查看,可以
看到 sys 破解出来的密码为 97396900965239D0,。如图 4、图 5 所示
图 4
图 5
1.6 打开 cain,在 Cracker 下选择 Oracle Hashs,先在空白处单击一下鼠标,再单击“+”添
加信息,用户名为 sys,hash 值为 97396900965239D0,单击“OK”。如图 6 所示
图 6
1.7 右键添加的条目,选择“Dictionary Attack”-->“UPPRECASE Hashs”。如图 7 所示
图 7
1.8 在 Dictionary 下方的空白处右键,选择“Add to list”,添加字典文件后,单击“Start”
按钮,开始破解。如图 8 所示
图 8
1.9 破解出 sys 的密码为 Simplexue123(这里显示的时候不区分大小写)。如图 9 所示
图 9
二、分析抓取的破解数据包,理解 oracle 数据库的破解过程
2.1 Wireshark 停止抓包,过滤端口号为 1521 的数据包。如图 10 所示
图 10
2.2 下面我们对其中一个过程的数据包进行分析。首先查看第一个数据包,源地址为 192.168.1.
2,源端口为 1357,目的地址为 192.168.1.3,目的端口为 1521,TCP 序号为 SYN,且置为
1,表示客户端 192.168.1.2 向服务器 192.168.1.3 发送了一个连接请求报文。如图 11 所
示
图 11
2.3 再查看第二个数据包,192.168.1.3 收到请求后确认联机信息,向 192.168.1.2 发送数据包,
SYN 置为 1,ACK 也置为 1。如图 12 所示
图 12
2.4 查看第三个数据包,主机 192.168.1.2 收到数据包后检查无误,向 192.168.1.3 发送数据包,
ACK 置为 1,至此三次握手完毕,连接建立。如图 13 所示
图 13
2.5 在 TNS(Transparent Network Substrate Protocol)协议中,每个 TNS 完整数据都包含一
个通用包头,说明接受数据的长度及其相关校验和解析的信息,类型 type 的值为 1 时,即
Connect(1),表示请求连接。如图 14 所示
图 14
其中,(DESCRIPTION=(CONNECT_DATA=(SERVICE_NAME=orcl)(CID=(PROGRAM=C:\Documents?and?
Settings\Administrator\????\tools\oracle\sqlplus.exe)(HOST=ADMIN-FAD0D3E1E)(USER=Ad
ministrator)))(ADDRESS=(PROTOCOL=TCP)(HOST=192.168.1.3)(PORT=1521)))描述了客户端登
录服务端时的连接信息。
2.6 当 type 的值为 11,即 Resend(11)时,表示重新发送,即服务器要求客户端再发送一次连接
请求。如图 15 所示
图 15
2.7 客户端再次发送连接请求。当 type 的值为 2,即 Accept(2)时,表示接受。如图 16 所示
图 16
2.8 当 type 的值为 6,即 Data(6)时,表示数据,即进行着数据的传输。继续往下分析数据包,
我们找到相关的登录请求信息数据包,我们发现无法看到明文密码,因为 oracle 的登录认
证过程密码是加密的,无法查看。如图 17 所示
图 17
2.9 登录失败信息如下。如图 18 所示
图 18
2.10 在抓取的破解数据包中,有这样一个数据包,通过分析其,可以发现这是登录成功时显示
的信息,说明破解出了 sys 的密码,登录成功了。如图 19 所示
图 19
2.11 登录成功后调用 selectpassword.sql 脚本,执行 SQL 语句“select name, password f
rom sys.user$;”,抓取出在 SYS.USER$表中的其他用户的哈希值 ,并将结果保存在 t
hepasswordsare.txt 文件中,最后退出程序。如图 20 所示
图 20
2.12selectpassword.sql 脚本内容如下。如图 21 所示
图 21
2.13 所以,只要在破解过程中,有一个密码可以登录成功,我们可以获得所有的用户名及其密
码 hash 值,再使用 cain 来破解密码。thepasswordsare.txt 的内容如下。如图 22 所示
图 22
三、查看日志
3.1 切换到 Windows server 2008 R2 上,打开 cmd,使用命令“sqlplus sys/Simplexue123 as
sysdba”进入 oracl。在 oracl 中,使用命令“show parameter audit;”查看和审计相关
的主要参数,audit_sys_operations 默认为 false,当设置为 true 时,所有 sys 用户(包
括以 sysdba,sysoper 身份登录的用户)的操作都会被记录,audit trail 不会写在 aud$表
中,这个很好理解,如果数据库还未启动 aud$不可用,那么像 conn /as sysdba 这样的连
接信息,只能记录在其它地方。如果是 windows 平台,audti trail 会记录在 windows 的
事件管理中,如果是 linux/unix 平台则会记录在 audit_file_dest 参数指定的文件中。所
以本实验审计 sys 的信息时使用事件查看器进行查看。如图 23 所示
图 23
3.2 在 Windows server 2008 R2 上刷新“应用程序”日志。如图 24 所示
图 24
3.3 发现有多条新增多条应用程序日志信息,查看某条信息记录。如图 25 所示
图 25
3.4 下面简略说明一下这条信息所包含的内容。
Audit trail: LENGTH : '170' ACTION :[7] 'CONNECT' DATABASE USER:[3] 'SYS' PRIVILE
GE :[4] 'NONE' CLIENT USER:[13] 'Administrator' CLIENT TERMINAL:[5] 'HOSTA' STATUS:
[4] '1017' DBID:[10] '1471216449' .
其中,Audit trail 表示审计跟踪,由该条信息可知为连接数据库信息,连接用户名为 SYS,客
户端用户为 Administrator,连接终端为 HOSTA(即 Windows server 2003 的主机名),STATUS:
[4] '1017'表示连接失败。如图 26 所示
图 26
3.5 在查看这些消息记录时,发现有一条记录与其他的记录相比,略有不同,STATUS:[1] '0'表
示数据库连接成功。在短时间内有大量的 oracl 数据库连接审计信息,在大多数连接失败
的信息中,有连接成功的记录,我们有理由相信数据库遭受到了暴力破解攻击。这时我们
就需要采取应对的应对措施。例如关闭数据库、更改数据库连接密码、限制登录连接次数
等等。如图 27 所示
图 27
3.6 再次清空“应用程序”日志。在 Windows server 2003(192.168.1.2)上,在 cmd 中输入命
令“sqlplus sys/Simplexue123@192.168.1.3:1521/orcl as sysdba”,使用破解出来
的密码进行登录,登录成功。如图 28 所示
图 28
3.7 在 Windows server 2008 R2(192.168.1.3)上查看“应用程序”日志,发现其信息与我们之
前猜测的一样,为登录成功的审计日志信息。如图 29 所示
图 29
3.8 Oracle 11g 中为了防止暴力破解数据库中用户的密码,提供了一种常见手段:延长失败尝试
响应。这种手段的策略是:在连续使用错误密码反复尝试登录时,从第四次错误尝试开始,
每次增加 1 秒的延迟,最长延迟目前是 10 秒。使用这种手段可以相对比较有效的防治用户
密码的暴力破解(能够使用到这种手段的前提是 FAILED_LOGIN_ATTEMPTS 参数设置的足够
大或无限大,否则用户密码超过 10 次的错误尝试之后该用户将被锁定。所以本实验为了演
示效果较佳,使用的密码文件的密码数较少,便于破解。
