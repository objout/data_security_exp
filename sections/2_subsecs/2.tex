\subsection{实验目的-了解 iptables 中常见的一些相关名词和术语}
了解 iptables 中常见的一些相关名词和术语.
%
\subsection{实验原理}
\begin{itemize}
  \item iptables 是与 Linux 内核集成的 IP 信息包过滤系统。如果 Linux 系统连接
    到因特网或 LAN、服务器或连接 LAN 和因特网的代理服务器,则该系统有利于在
    Linux 系统上更好地控制 IP 信息包过滤和防火墙配置。
  \item 防火墙在做信息包过滤决定时,有一套遵循和组成的规则,这些规则存储在
    专用的信息包过滤表中,而这些表集成在 Linux 内核中。在信息包过滤表中,规
    则被分组放在我们所谓的链(chain)中。而 ``netfilter/iptables'' IP 信息包过
    滤系统是一款功能强大的工具,可用于添加、编辑和移除规则。
\end{itemize}
%
\subsubsection{iptables 防火墙}
\begin{itemize}
  \item Netfilter/Iptables(以下简称 iptables)是 Unix/Linux 自带的一款优秀
    且开放源码的完全基于包过滤的防火墙工具,它的功能十分强大,使用非常灵活,
    可以对流入和流出服务器的数据包进行很精细的控制。特别是它可以在一台非常
    低的硬件配置下跑得非常好。
  \item Netfilter/Iptables 的最大优点是它可以配置有状态的防火墙,这是老一
    辈 ipfwadm 和 ipchains 等以前的工具都无法提供的一种重要功能。Iptables 主
    要工作在 OSI 的二、三、四层,即数据链路层、网络层和传输层,如果重新编译
    内核,也可以支持 7 层控制。
  \item Iptables 防护墙由两个组件 netfilter 和 iptables 组成。
  \item netfilter 组件也称为内核空间(kernelspace),是内核的一部分,由一
    些信息包过滤表组成,这些表包含内核用来控制信息包过滤处理的规则集。
  \item iptables 组件是一种工具,也称为用户空间(userspace),它使插入、修
    改和除去信息包过滤表中的规则变得容易。
\end{itemize}
%
\subsubsection{iptables 名词和术语}
\begin{itemize}
  \item 容器:在 iptables 中,容器用来描述包含或者属于的关系。
  \item Netfilter:Netfilter 是表(tables)的容器。
  \item 表(tables):表(tables)是链的容器,即所有的链(chains)都属于对
    应的表(tables)。
  \item 链(chains):链(chains)是规则(policy)的容器。
  \item 规则(policy):规则(policy)是 iptables 一系列过滤信息的规范和具体方法。
\end{itemize}
我们可以用一个通俗易懂的例子表现这几个术语之间的关系。
\begin{table}[H]
  \begin{center}
    \begin{tabular}[c]{llll}
      \hline
      Netfilter & tables(表) & chains(链) & policy(规则) \\
      一栋楼 & 楼里的房间 & 房间里的柜子 & 柜子里的衣服摆放规则 \\
      \hline
    \end{tabular}
  \end{center}
\end{table}
%
